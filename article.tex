% % \section{Introduction}

% % Musical instruments provide an interesting problem for conservation as they are simultaneously aesthetic objects and tools for composing music.
% % This is  of course an over simplification as the function of musical instrument and in part this article will seek to explore the nature and a re-framing of their conservation. 
% % Also, discussed below is the details of a new exhibition at Museo San Colombano which aims at reintroducing the experience of playing a harpsichord, the motivations behind the project, it's technical, and potential routes for expansion.

% % Much hay has been made in recent decades for  values-based conservation versus living heritage \cite{poulios_moving_2010}.
% % Poulios argues that values-based approaches to heritage conservation regard the pass as `dead` with a discontinuity between the past and present \cite{poulios}. 
% % Instead a `living heritage` is argued where the heritage in question retains it's original function. Polios in this  case is discussing heritage sites but much of the same philosophy can be applied to  heritage objects. 
% % It is a false dilemma to suggest that there exists only a choice between living or values-based heritage.
% % By their definition, living things expire. If a heritage is never permitted to `die' then it is more akin to `un-dead' than living.
% % I argue its is therefore not a choice, but a necessity that heritage incorporate both values-based as well as living heritage approaches when applied to musical instruments. 
% % We may even need to go a little further, but that thread is to be discussed in a later section.


% % \subsection{A New Value for Conservation}

% % Borrowed from \cite{avrami_values_2000} is a table of values which would guide justification for conserving an object.

% % \begin{table}[h]
% %     \centering
% %     \begin{adjustbox}{width=\textwidth}
% %         \begin{tabular}{llll}
% %             \toprule
% %             \textbf{Art History} & \textbf{ICOMOS Australia} & \textbf{Economics} & \textbf{English Heritage} \\
% %             Alois Reigl & Burra Charter & Bruno Frey &  \\ 
% %             1902 & 1998 & 1997 & 1999 \\ \hline
% %             Age& Aesthetic& Monetary & Cultural\\ 
% %              Historical &  Historic&  Option &  Educational and academic \\ 
% %              Commemorative& Scientific& Existence & Economic\\ 
% %              Use&  Social (including spiritual, 
% %     political, national, other 
% %     cultural)&  Bequest &  Resource \\ 
% %              & & Prestige & Recreational \\
% %              & & Educational & Aesthetic\\
% %         \end{tabular}
% %     \end{adjustbox}
% %     \caption{Comparative Analysis of Value Systems}
% %     \label{tab:values_comparison}
% % \end{table}

% % In the case of musical instruments, and I would argue all heritage, what is missing explicitly is `instructional value', that is the capacity with which the heritage object has to instruct within a given practice.
% % In the same way a painting or a sculpture can go some way to instruct the observer on practice from the `witness marks' left by creative and fabrication processes, an instrument musical instrument can also instruct a luthier or a musical performer alike.

% % It could be argued that `instructional value' is a subset of educational / academic value.
% % When educational value can also be simply historical knowledge, I argue that we should be more deliberate in ascribing value.
% % There exists a clear division between knowledge of facts and knowledge of practice, of skill.
% % Part of the trouble we find ourselves in with musical instrument conservation can in part be attributed to the lack of intentional division between the educational and the instructional.

% % For musical, much of the focus is placed on the `authenticity' of the performance \cite{laurenson_authenticity_2006, davies_authenticity_2001}. 
% % Putting aside the problems with the term `authenticity' for a moment, I would argue that the audience experience dominates the discussion of musical performance.
% % As an example, if a performer rehearses a piece of music 50 times to play to an audience of 50 people, who heard the music more, the audience or the musician?
% % I hold the position that it is the musician the had experience the music more and therefore their experience of rehearsing and playing the instrument has a greater value as instruction than audience does for cultural or educational value.
% % Not only is the performer is instructed but their performance is also influenced by the instrument they have used.
% % This is especially relevant if the instrument in question is a modern Steinway or an historical harpsichord.
% % The performer will be influence by, as Levinson puts it with resect to the use of modern or period appropriate clarinets:

% % \begin{quotation}
% %     the matrix of the gestural repertoires and sonic capacities of just the instruments for which the piece was conceived
% % \end{quotation}
% % \begin{flushright}
% % -- \cite[][]{levinson_music_1990}
% % \end{flushright}

% % It is perhaps ``a shock to be reminded'' once again that in fact ``the medium is the message'' \cite{mcluhan_understanding_1964}.

% % Typically the museum does not invite visitors to play their instruments with reckless abandon, rather they are still presented behind a `red velvet cord', real or imaginary, to be seen and not touched.
% % How do we then reintroduce the playing experience which encodes the vibro-tactile and kinaesthetic cues that cannot be retrieved simply by looking?

% % \subsection{Instruments and conservation}

% % How to then conserve instruments if they are finite?  Maintaining an instrument presents an interesting Ship of Theseus problem or, if you are of certain age and from the United Kingdom, a Trigger's Broom problem. Theseus's ship with its boards replaced one-by-one, Trigger's broom with its countless new brush heads and handles, both at some point would cease to be `the original'. 
% % The same though experiment can be applied to a musical instrument. 
% % How many keys, vales, plectra, bridges \&c. of an instrument need to be replaced before it is no longer original? 
% % For Laurenson, the answer depends on the value and the `function' and ``change is assessed by considering the significance'' of a component \cite{laurenson_management_2005}.
% % The problem with this approach is that it simply `kicks the can down the road'. 
% % In fact, this is highlighted by Laurenson in the case of the Walkman in Angus Fairhurst’s piece \textit{Gallery Connections}, whose high value and obsolescence dictates that the original should be kept and its function replicated in another manner \cite{laurenson_management_2005}.
% % This is presented as simply the nature of the business, rather than a process that needs some reflection.
% % For historical musical instruments, the can is already down the road and they do not have the benefit of delaying the inevitable.
% % Some of Laurenson's oversights can be traced back by taking misguided cues from Davies \cite{davies_authenticity_2001} who misplaces an over emphasis on authenticity with respect to substitution in musical performance and period instruments.

% % \begin{quotation}
% % Where the change in instrumentation, and adaptations made to the music in the light of this, are significant, the result is a work transcription that is distinct from its model. One does not transcribe a work merely by crossing out the word ‘harpsichord’ on the score and replacing it with ‘piano’ (Davies 1988a), however. Neither does the substitution of the modern cello for its Baroque cousin result in a work transcription. 
% % \end{quotation}

% % Davies presents some false equivalence between the differences of harpsichord and pianoforte, and those of the baroque violin and it's modern counter part.
% % The first set in a difference of kind with the harpsichord and pianoforte sharing a keyboard but completely different mechanisms for string excitation. 
% % The second set is a very much a comparison of degree. The `baroque-violin' is very much an unfixed design with variations in scale length and dimensions, but it and it's modern counterpart require excitation of the string using the same physics.
% % This may seem tangential, but it suggests that Davies does not fully comprehend the problem at hand and it's variation across instruments.

% % \begin{quotation}
% % It is relevant to add a second observation to this first: period instruments, when played in the style of the day, sound different from modern ones, but, as well, they sound better. They make clearer or more salient features of the work that are aesthetically significant. The quotation at the beginning of this chapter from Ton Koopman illustrates the point.
% % \end{quotation}

% % To allow some leniency to Davies, there have scientific been studies exploring these ideas, but they do result in Davies assessment here being objectively incorrect.
% % Recent studies by Fritz have shown \cite{fritz_player_2012, fritz_soloist_2014, fritz_listener_2017}, old violins are not a `fine wine' to whom age serves well. 
% % It is not a stretch to apply the same conclusions to the rest of the string family, being as they are some configuration of tone wood and strings under tension \cite{carcagno_guitar_2017}.
% % Where this all becomes a problem is when Davies incorrect ideas are used as the basis for conservation policy.
% % Perhaps now is a good time for us to reevaluate how we approach the problem of musical instrument conservation.

% % \subsection{Funereal Heritage}

% % Given what we now know from the Fritz studies, and modifying Laurenson's definition of ``time-based media installations'' \cite{laurenson_management_2005}, we could define musical instruments as a `time-based performance tool'. Taking the entire lifespan of heritage instrument, this means that there will be a period where it can be considered `living' but that it will inevitably transition transition to a static object. This is in part a call for the heritage sector to be more intentional as to how that transition is carried out.

% % Holzer et al. suggest a different apporach to simply reproducing or refurbishing instrumets and advicate instead for an approach more akin to historical reenactment \cite{holzer_imperfect_2025}. The keyboard for San Colombano treads somewhere between reproaduction and reencactment.


% % \section{Cutting the Velvet Cord}

% % The metaphor of cutting the velvet cord originated with project by McAlpine \cite{mcalpine_sampling_2014}. That project along with that of tromba marina project by \cite{baldwin_tromba_2016} influenced some of the philosophy and design process of the San Colombano Keyboard.

% % McAlpine discusses a case similar to the Tagliavini} collection in his examination of the Benton Fletcher Collection at National Trust Fenton House \cite{mcalpine_sampling_2014}. When these instruments were donated, Benton Fletcher stipulated that they remain playable and should continue to be maintained for tuition and public performance. A large sampling campaign was conducted, and a custom MIDI interface was designed to fulfil this requirement while preserving the original instruments' integrity. The MIDI controller, comprising two commercially available keyboards mimicking the two-manual harpsichord layout, was used by visitors to trigger the instrument samples recorded with tailored strategies for each. However, user tests identified a limitation: the commercially available weighted keys failed to provide an authentic sense of interacting with a historical plucked keyboard instrument \cite{mcalpine_sampling_2014}. 

% % On the other hand, the ``Tromba Moderna'' project \cite{baldwin_tromba_2016}, a previous NIME initiative, approached the issue of musical heritage playability by recreating and augmenting a replica of a historical tromba marina. A piezo transducer was connected to a sound synthesis engine and a driver within the instrument to simulate the expected vibrations of a historical tromba marina. 

% % Technological advances have transformed how museums document, present and interpret their collections. Immersive experiences are realised through tools such as 3D printing and virtual reality \cite{allard_use-of-hand-held_2005,wachowiak_3d-scanning_2009,music_3d-printed_2024,kuzminsky_three-dimensional_2012,schaich_from_2007}. These technologies form a kind of experiential authenticity, enabling encounters that evoke the past's sensory, emotional, and intellectual essence \cite{trant_when_1999}. However, as Pine and Gilmore note \cite{pine_museums_2007}, achieving authenticity requires museums to navigate the delicate balance between preservation and meaningful engagement—a challenge that is particularly evident in the case of historical musical instrument collections \cite{mcalpine_sampling_2014}. Preservation concerns often limit direct interaction, reducing these artefacts to static displays.

% % The Tagliavini} Collection in Bologna \cite{meer_collezione_2007}}, housed at Museo San Colombano - Genus Bononiae} and renowned for its historical keyboard instruments, exemplifies this dilemma. With over fifty early keyboard instruments, primarily early plucked stringed keyboards of Italian origin, the collection stands out as a valuable resource for musicologists, organologists and musicians alike. Preserving the instruments' authenticity was the cornerstone of Ferdinando Tagliavini}’s vision. This guiding principle led him to collect instruments that could be restored to their playing condition after minimal intervention. 

% % However, the delicate mechanisms of these instruments and their historical significance mean they are only played under strict conditions—by experienced historical keyboard performers or young musicians under strict supervision from the curator. 
% % To enhance accessibility and engagement, the museum commissioned the replica (Figure \ref{fig:teaser}) of a historical keyboard built in the tradition of the Italian school, which is the subject of this paper.

% % The keyboard targets all visitors to the museum, regardless of playing ability, and can be used without special permission.
% % Jacks that pluck two choirs of muted strings, across 49-keys, are used to generate MIDI messages that are sent to a connected computer for audio synthesis. 
% % The interface is presently linked to a commercial software sampler; however, the ultimate aim is to make the sound of instruments in the collection that can no longer be maintained in playable condition accessible. Museum visitors are invited to play the interface and listen through a pair of headphones. This work outlines the technological aspects of the interface's construction and offers reflections on its role within the Tagliavini} Collection and potential application within the wider musical instrument museum context. 

% % The optical sensing technique for the keyboard is adapted from a similar project on the piano by McPherson \cite{mcpherson_portable_2013}. As such, this project extends the longevity of the results beyond the scope envisaged in the original works, enhancing their sustainability, as discussed by Masu \emph{et al.} \cite{masu_the-o-in-nime:_2023}. The intended use of the keyboard through meaningful interaction with a museum exhibit is where the novelty of this work lies, rather than solely in its technological development. 

% % \section{Design Principles}\label{design}

% % The design of the augmented replica keyboard for the Museo San Colombano - Genus Bononiae} was guided by four principal constraints:

% % \begin{itemize}
% %     \item \emph{Faithfulness}. The keyboard mechanism had to ensure fidelity to its authentic operation. The electronics system would not seek to `fix' or `improve' the limitations of the original design.     
% %     \item \emph{Robustness and reliability}. The system needed to accommodate frequent use by museum visitors and allow for straightforward maintenance by staff without requiring specialised technical expertise.
% %     \item \emph{No visible electronic components}. To preserve the visual integrity of the exhibit, all electronic components had to remain concealed except for a pair of headphones and a small display for audio parameter adjustments.
% %     \item \emph{Reduced use of space}. As museums often face space limitations, the instrument needed be compact enough not to compromise space for exhibition of the permanent display. 
% % \end{itemize}

% % The final design, shown in Figures \ref{fig:teaser} and \ref{fig:details}, is a 49-key, two-register harpsichord keyboard replica. Since the Tagliavini collection} hosts primarily early plucked stringed keyboards of Italian origin, the keyboard layout was deliberately modelled after early Italian harpsichords, leveraging the human tendency to be influenced by visual elements when making musical judgments \cite{tsay_sight_2013}. This effect is particularly relevant in musical instruments, as demonstrated by studies conducted by Fritz \emph{et al.} \cite{fritz_player_2012, fritz_soloist_2014, fritz_listener_2017}. Instead of attempting to mitigate this influence, the design embraces it. This phenomenon, akin to a `musical instrument McGurk effect’ \cite{mcgurk_hearing_1976,politzer_mcgurk_2019}, is one of the reasons why the electronics have been concealed from view. The aesthetics of the interface enhance its likelihood of being perceived as an `authentic' musical instrument. Given the extensive research into modelling and recreating piano action \cite{cadoz_a-modular_1990, gillespie_haptic_1996, timmermans_multibody-based_2020}, the visual component may provide the final persuasive element necessary for acceptance, similar to the way visual perception affects judgments of musical performance \cite{tsay_sight_2013}.

% % \subsection{Materials and Construction}
% % The keyboard was designed to replicate the tactile and aesthetic sensations of playing a historical Italian harpsichord. Traditional materials were used, including walnut for the wrest plank, chestnut for the key levers, boxwood and ebony for key covers, and cypress for the case and soundboard. The 98 jacks were made from beech, fitted with brass springs and natural seagull feather plectra. The design was inspired by the 1547 Alessandro Trasuntino} harpsichord at the San Colombano Museum}, considered one of the first prototypes to have been conceived with two sets of 8-foot strings. An exception was made for the short octave --- the assigning of common keys in the first octave instead of a chromatic scale --- typically found in Italian harpsichords, which was replaced by a standard octave layout to more easily accommodate the interaction with commercial sample libraries. 

% % The rectangular poplar frame also deviates from the traditional logarithmic (Figure \ref{fig:log-harp-comp}) form to allow for the installation of the electronic sensors, as visible in Figure \ref{fig:49-key-bottom}, without compromising the visual or tactile authenticity. Two string choirs, crafted from yellow brass wire and anchored with wrought iron pins, were tensioned to replicate authentic plucking resistance. Felt strips were added to dampen vibrations. The result is an interface that combines the mechanical action of keys with synthetic sound generation, preserving a real harpsichord's tactile qualities.

% % A further objective, set forth by the authors, was to ensure reproducibility by committing to an open source approach for all outputs of the project. The commitment to open sourcing encompassed all aspects of the system, including hardware schematics, firmware, and calibration data. Cost-effectiveness was also a central consideration.

% % \section{Discussion}\label{context}

% % The exhibition opened February 2025 and is hosted in the \emph{Oratory} above the museum's main hall, an exceptional testimony to the Bolognese art, decorated by the finest students of the Carracci and presenting a series of frescoes covering most of the walls and ceiling. The room hosts unique examples of the Italian Renaissance building tradition, including the 1547 harpsichord and the 1540 spinet by Alessandro Trasuntino}. Visitors often visit the room with the same caution and respect typical of worship spaces. Among the visitors interviewed, no comments were made that the exposed headphones and touchscreen affected such an experience negatively, though a more definitive answer will come as more feedback is collected after launch. Pictures of the keyboard hosted in the \emph{Oratory} are shown in Figure \ref{fig:oratory}.

% % Since the opening of the exhibition, there have been two rounds of preliminary feedback collected. The first from a pool of 20 people -- consisting of staff, surveillance personnel and visitors who were given training on the system -- and expert feedback from Catalina Vicens} and Roberto Livi}.
% % Visitor feedback was collected through short informal interviews and primarily to identify technical problems.
% % The keyboard shows promise in enhancing the museum experience as comments about a ``sense of disconnect'' found in the Benton Fletcher Collection \cite{mcalpine_sampling_2014} were absent in this initial feedback.
% % The second round a survey was conducted on a group of approximately 50 PhD students on the subject of the wider San Colombano exhibition and not solely the keyboard. The survey was in a five-point scale format the results of which are still being collected. 
% % There were some unexpected issues raised from casual observations observation of the visitor group.
% % There were around 10 visitors who, after observing the keyboard, did not play it. When asked why the responded similarly in that they did not think they were allowed to touch it. For the rest of collection this is true and although there was signage -- admittedly small -- that invited visitors to play, it did not appear to be enough to convince some visitors other wise. This could be considered in compliment for accuracy in the aesthetics of the keyboard in that it was being accepted on the same level as the other instruments in the room. This does pose the interesting challenge, however, of how to encourage visitors to break out of the trained behaviour of simply observing and not interacting. 

% % Expert feedback has highlighted limitations, including imprecise key calibration, which creates a temporal disconnect between the tactile plucking sensation and sound onset. Additionally, the commercial sample library, while allowing the selection of registers such as 8', 4', and their combinations, restricts the output to a single sample, regardless of the number of string choirs controlled. Future iterations could address these limitations through technical improvements, such as refined calibration and developing a custom digital audio instrument to accommodate multiple registers.

% % By integrating historical keyboard-building traditions with digital augmentation, it offers a possibility to extend the practical lifespan of musical heritage while maintaining the tactile qualities of the original instrument. Rather than prioritising technical novelty, the project demonstrates how digital interventions can support multi-modal interaction with a shared musical heritage, ensuring its continued relevance in contemporary museum contexts.

% %----

% Musical instruments pose a unique challenge for conservation, serving both as aesthetic objects and as tools for musical practice. This is, of course, an oversimplification. The function of a musical instrument is multifaceted, and this article aims to reframe how we consider their conservation. In doing so, it will also discuss a recent exhibition at Museo San Colombano that seeks to reintroduce the embodied experience of playing a harpsichord—exploring the project’s motivations, technical execution, and potential for future expansion.

% In recent decades, considerable debate has emerged around values-based conservation versus approaches rooted in living heritage \cite{poulios_moving_2010}. Poulios argues that values-based conservation often treats the past as ‘dead’, enforcing a discontinuity between past and present \cite{poulios_moving_2010}. By contrast, living heritage approaches advocate for preserving an object’s original function. While Poulios focuses primarily on heritage sites, this philosophy can be equally applied to heritage objects.

% However, framing this as a dichotomy—values-based or living heritage—is misleading. Living things, by definition, expire. If heritage is never allowed to "die," it becomes something undead rather than truly living. Therefore, the conservation of musical instruments necessitates an integration of both values-based and living heritage perspectives. In fact, we may need to go even further, a possibility explored later in this article.

% Drawing on the value frameworks outlined in \cite{avrami_values_2000}, we compare the varied justifications for conservation below:

% \begin{table}
%     \centering
%     \begin{adjustbox}{width=\textwidth}
%         \begin{tabular}{llll}
%             \toprule
%             \textbf{Art History} & \textbf{ICOMOS Australia} & \textbf{Economics} & \textbf{English Heritage} \\
%             Alois Riegl & Burra Charter & Bruno Frey &  \\ 
%             1902 & 1998 & 1997 & 1999 \\ \hline
%             Age & Aesthetic & Monetary & Cultural \\ 
%             Historical & Historic & Option & Educational/academic \\ 
%             Commemorative & Scientific & Existence & Economic \\ 
%             Use & Social (spiritual, political, national, cultural) & Bequest & Resource \\ 
%             & & Prestige & Recreational \\
%             & & Educational & Aesthetic \\
%         \end{tabular}
%     \end{adjustbox}
%     \caption{Comparative Analysis of Value Systems from \cite{avrami_values_2000}}
%     \label{tab:values_comparison}
% \end{table}

% What is notably absent, especially regarding musical instruments, is instructional value—the potential for an object to inform practice. Just as the wear marks on a sculpture or painting can hint at their creation process, a historical instrument can educate both luthiers and performers.

% One might argue that instructional value falls under educational or academic categories. Yet educational value often implies the transfer of factual knowledge, whereas instructional value refers to tacit knowledge—skills, gestures, and embodied practice. The lack of this distinction in conservation discourse has contributed to ongoing challenges in preserving musical instruments.

% The focus in musical heritage often centres on the 'authenticity' of performance \cite{laurenson_authenticity_2006, davies_authenticity_2001}. However, this frequently privileges audience experience over the performer's. For instance, if a musician rehearses a piece fifty times for an audience of fifty, who has truly \textit{experienced} the music more? The performer, through repeated interaction, gains deeper instructional value. Moreover, the instrument itself shapes the performance. As Levinson notes:

% \begin{quotation}
%     the matrix of the gestural repertoires and sonic capacities of just the instruments for which the piece was conceived
% \end{quotation}
% \begin{flushright}
% -- \cite[][]{levinson_music_1990}
% \end{flushright}

% It is a useful reminder that, indeed, “the medium is the message” \cite{mcluhan_understanding_1964}.

% Most museums do not allow their instruments to be played freely. They remain behind a physical or symbolic velvet rope—seen, not heard. So how do we reintroduce the kinaesthetic and vibro-tactile knowledge that is lost when instruments are only observed?

% If instruments are inherently finite, how can we meaningfully conserve them? This is analogous to the Ship of Theseus—or, to UK audiences, Trigger's Broom. If every part is eventually replaced, does the object remain original?

% Laurenson argues that value and function should guide decisions, with changes assessed according to significance \cite{laurenson_management_2005}. However, this approach often defers rather than solves the problem. In fact, Laurenson's own case study of Angus Fairhurst’s \textit{Gallery Connections}demonstrates this. The Walkman's obsolescence and value necessitate preserving the original while emulating function elsewhere.

% The same reasoning should be applied to musical instruments. We can no longer afford to postpone such decisions. Some of Laurenson's limitations stem from over-reliance on Davies \cite{davies_authenticity_2001}, who arguably overemphasises the role of authenticity and makes problematic equivalences:

% \begin{quotation}
% One does not transcribe a work merely by crossing out the word ‘harpsichord’ and replacing it with ‘piano’... Neither does the substitution of the modern cello for its Baroque cousin result in a work transcription.
% \end{quotation}

% This comparison is misleading. Harpsichords and pianos differ in kind—one plucks, the other strikes. By contrast, Baroque and modern violins differ more in degree. These distinctions are crucial and often ignored in conservation policy.

% Davies goes further:

% \begin{quotation}
% Period instruments, when played in the style of the day, sound better. They make more salient the aesthetically significant features of the work.
% \end{quotation}

% Yet this claim is not supported by evidence. Studies by Fritz \cite{fritz_player_2012, fritz_soloist_2014, fritz_listener_2017} show that listeners and performers cannot reliably distinguish between old and new violins. The mystique of ‘fine wine’ aged instruments may be overstated.

% When such misconceptions inform conservation policy, problems arise. It is time to reconsider our approach.

% Building on Laurenson’s term “time-based media installations” \cite{laurenson_management_2005}, we might describe musical instruments as time-based performance tools. Their life spans include a 'living' phase—when they are playable—and a later static phase, when they no longer function as originally intended.

% Holzer \emph{et al.} suggest that rather than restoring or reproducing instruments, we should treat them more like actors in historical re-enactments \cite{holzer_imperfect_2025}. The San Colombano keyboard occupies a middle ground between reproduction and re-enactment, allowing interaction while preserving the original's integrity.

% The metaphor of “cutting the velvet cord” comes from McAlpine’s project \cite{mcalpine_sampling_2014}, alongside Baldwin’s \cite{baldwin_tromba_2016} work on the tromba marina. Both influenced the San Colombano replica keyboard design.

% McAlpine worked with the Benton Fletcher Collection, where donor stipulations required the instruments remain playable. A MIDI sampling system was created to allow public engagement without risking the originals. However, the lack of haptic authenticity in the keyboard interface limited user experience.

% By contrast, Baldwin’s “Tromba Moderna” recreated a historical instrument and embedded vibration transducers to simulate tactile response—providing more convincing feedback.

% Technology is increasingly used in museums to document and reinterpret artefacts \cite{allard_use-of-hand-held_2005,wachowiak_3d-scanning_2009,music_3d-printed_2024,kuzminsky_three-dimensional_2012,schaich_from_2007}. As Pine and Gilmore note \cite{pine_museums_2007}, creating authentic experiences requires balancing preservation with engagement.

% The Tagliavini Collection at Museo San Colombano exemplifies this challenge. Comprising over 50 historic keyboards, it was curated by Ferdinando Tagliavini to prioritise restorability and playability. However, access remains restricted.

% To address this, the museum commissioned a replica that mimics the physical action of a harpsichord while generating sound digitally. Visitors are invited to play and listen through headphones. The keyboard transmits MIDI data via mechanical jacks plucking muted strings—a hybrid interface that blends tactile realism with technological accessibility.

% This project follows the sensing approach pioneered by McPherson \cite{mcpherson_portable_2013} and adopts an open-source philosophy for reproducibility and sustainability \cite{masu_the-o-in-nime:_2023}.

% \section{Design Principles}\label{design}

% The replica was built with four constraints:

% \begin{itemize}
%     \item \textbf{Faithfulness}: Maintain the original playing mechanics; avoid “improvements.”
%     \item \textbf{Robustness}: Designed for frequent, non-specialist use.
%     \item \textbf{Concealed Electronics}: Preserve visual authenticity.
%     \item \textbf{Compact Form}: Accommodate limited museum space.
% \end{itemize}

% Modelled after early Italian harpsichords, the keyboard contains 49 keys and two string choirs. Traditional materials—walnut, chestnut, boxwood, cypress—were used. Aesthetic fidelity was essential, as visual cues strongly influence musical judgment \cite{tsay_sight_2013, fritz_player_2012}. The interface intentionally embraces this bias.

% Inspired by the 1547 Alessandro Trasuntino harpsichord, the replica retains traditional craftsmanship while allowing integration of sensors and electronics. Some compromises were made, such as replacing the short octave with a chromatic layout for compatibility with digital sampling.

% The result is a tactile interface that preserves the feel of a historical harpsichord while enabling sound synthesis.

% The exhibition launched in February 2025, housed in the museum’s \emph{Oratory}, a richly decorated Renaissance space. Despite concerns, no visitors reported the touchscreen or headphones as diminishing the experience.

% Two feedback rounds have been completed. In the first, informal interviews with staff and musicians noted positive engagement and few technical issues. In contrast to earlier projects like Benton Fletcher’s, no participants reported a sense of disconnection.

% A second survey (50 PhD students) gathered broader exhibition impressions. Interestingly, about 10 visitors hesitated to play the keyboard, assuming it was off-limits—perhaps a testament to its visual authenticity. Better signage may help overcome such hesitations.

% Expert feedback highlighted calibration issues and limitations in the commercial sample library’s register handling. Future work could include custom sound engines and refined sensor calibration.

% Ultimately, the project blends historical instrument building with digital augmentation to provide accessible, meaningful interaction with musical heritage. The aim is not just to preserve instruments but to preserve the experience of playing them.

% %%

Musical instruments present a uniquely challenging and fascinating problem within the field of conservation. Unlike many heritage objects, they occupy a dual status: they are both aesthetic artifacts of cultural and artistic significance and functional tools for producing music. This duality complicates conservation approaches, as efforts must balance preserving an instrument's physical integrity and historical authenticity with retaining its functional capability as a sound-producing device. This tension is central to understanding the broader philosophical and practical questions involved in the conservation of musical instruments, and it invites a re-examination of traditional frameworks to better encompass their hybrid nature.

The scope of this paper is twofold: first, to explore the nature of musical instruments as heritage objects and propose a re-framing of their conservation that acknowledges their unique characteristics; second, to present and analyse a recent project at Museo San Colombano in Bologna, Italy, where a new exhibition and an augmented replica harpsichord keyboard aim to reintroduce the experience of playing historical keyboard instruments. This project embodies the intersection of conservation philosophy, technical innovation, and museum practice, providing a practical case study on how living interaction with heritage instruments might be facilitated while safeguarding their material and cultural values.

Over recent decades, the discourse on heritage conservation has witnessed a significant shift, with a greater discussion around \emph{living heritage} and the static, preservable `dead' heritage \cite{poulios_moving_2010,smith_uses_2006}. 
%% seems that `dead heritage` is fundamentally negative
\footcite[\textit{`a handful dismissed the idea of heritage as a negative idea, noting for instance that heritage was ‘keeping that which aught to be alive dead.}' What about keeping that which aught to be dead alive][]{smith_uses_2006}
Poulios critiques values-based conservation for treating the past as something `dead,' an objectified moment frozen and isolated from the present, resulting in a discontinuity between past and present experiences \cite{poulios_moving_2010}. In contrast, the \emph{living heritage} paradigm emphasizes the ongoing, dynamic use and relevance of heritage, advocating for the preservation of not just physical artifacts but the practices, skills, and meanings that sustain them in contemporary life.

While Poulios' arguments are primarily directed at heritage sites, the underlying philosophy resonates deeply with the conservation of heritage objects, particularly musical instruments. The challenge is to transcend a false dichotomy that presents conservation as a choice between preserving static, decontextualized objects and enabling their continued use. The reality is more complex: living things—by definition—age, wear, and eventually perish. To treat heritage as perpetually living without acknowledging its material finitude risks creating an `undead' heritage, neither truly alive nor respectfully preserved.

I argue here that a holistic conservation framework for musical instruments must integrate both values-based and living heritage perspectives. Such a synthesis acknowledges the physical realities of ageing instruments while honouring their performative and instructional roles. Moreover, this integration opens pathways to novel conservation strategies that extend beyond current paradigms, a discussion that will be elaborated upon in subsequent sections.

% \subsection{A New Value for Conservation}

The justification for conserving heritage objects often rests upon a complex matrix of values. Avrami et al. \cite{avrami_values_2000} provide a comparative analysis of value systems that underpin conservation decisions across different disciplines and organizations, summarized in Table \ref{tab:values_comparison}.

\begin{table}[h]
    \centering
    \begin{adjustbox}{width=\textwidth}
        \begin{tabular}{llll}
            \toprule
            \textbf{Art History} & \textbf{ICOMOS Australia} & \textbf{Economics} & \textbf{English Heritage} \\
            Alois Reigl & Burra Charter & Bruno Frey &  \\ 
            1902 & 1998 & 1997 & 1999 \\ \hline
            Age & Aesthetic & Monetary & Cultural \\ 
            Historical & Historic & Option & Educational and academic \\ 
            Commemorative & Scientific & Existence & Economic \\ 
            Use & Social (including spiritual, political, national, other cultural) & Bequest & Resource \\ 
            & & Prestige & Recreational \\
            & & Educational & Aesthetic \\
            \bottomrule
        \end{tabular}
    \end{adjustbox}
    \caption{Comparative Analysis of Value Systems in Heritage Conservation \cite{avrami_values_2000}.}
    \label{tab:values_comparison}
\end{table}

In addition to those above, I propose as a crucial category for musical instruments and potentially all heritage objects that embody craft and practice: \emph{instructional value}. Instructional value refers to the capacity of a heritage object to convey knowledge of practice, technique, and skill intrinsic to its use and making. For example, a painting or sculpture may reveal insights into artistic methods through `witness marks', the traces of the artist’s hand or materials. Similarly, a musical instrument can serve as a pedagogical tool for luthiers, performers, and scholars, instructing through its construction, wear patterns, and interaction how music was historically produced and experienced.

Instructional value may superficially appear to be a subset of educational or academic value. However, I argue that this conflation obscures a critical distinction between \emph{descriptive knowledge} (knowing that) and \emph{procedural knowledge} (knowing how). Historical knowledge often privileges the former-—facts, dates, styles-—whereas instructional knowledge engages with embodied practice and tacit skills. This division is essential to resolving tensions in musical instrument conservation, where preserving the physical object alone cannot fully capture its performative essence without maintaining or reviving the skills and practices it embodies.

The discourse around musical instrument conservation frequently centres on the notion of \emph{authenticity}, particularly regarding historically informed performance \cite{laurenson_authenticity_2006,davies_authenticity_2001}. Setting aside the complexities and debates surrounding the meaning of `authenticity,' it is important to recognize that the audience's experience often dominates discussions of musical heritage. Yet, consider a scenario where a performer rehearses a piece repeatedly before an audience hears it once. Who experiences the music more deeply—the musician or the listener? The performer, through hours of embodied interaction, acquires a richer, more nuanced understanding, gained through tactile feedback and `gestural repertoires' \cite[][]{levinson_music_1990} intrinsic to playing the instrument.

Thus, the performer’s experience, informed by the instrument’s unique qualities, carries significant instructional and cultural value. The instrument shapes the performance as much as the performer shapes the music. This interplay is especially salient when comparing, for instance, a modern Steinway piano with a historical harpsichord, where differences in touch, sound production, and response create distinct musical experiences.

In other words, the instrument is not merely a medium but an active participant in the musical message, reinforcing McLuhan’s famous dictum that ``the medium is the message'' \cite[][]{mcluhan_understanding_1964}.

Conventional museum practice rarely permits visitors to handle or play historical instruments. Instead, they are typically cordoned off behind a -- real or imagined -- `red velvet cord,' displayed as objects to be seen but not touched. This approach excludes the experience of vibro-tactile and kinaesthetic cues encoded into the interface of the instrument. The challenge, then, is how to reintroduce embodied interaction with historical instruments in ways that respect conservation imperatives while restoring some degree of their living function.

% \subsection{Instruments and Conservation}

Conserving musical instruments also raises the philosophical puzzle akin to the \emph{Ship of Theseus}: if an object’s components are replaced one by one over time, at what point does it cease to be the original? This problem is mirrored in the British cultural reference of \emph{Trigger's Broom}, a seemingly identical yet continuously repaired artifact. For instruments, this question is concrete: how many keys, valves, plectra, bridges, or strings can be substituted before the instrument loses its historical identity?

Laurenson addresses this by suggesting that the significance and function of each component must guide decisions about conservation and change, emphasizing the value-based assessment of alterations \cite{laurenson_management_2005}. However, this approach risks deferring the problem without resolving the ontological tension—essentially `kicking the can down the road.’ For example, Laurenson discusses the Walkman in Angus Fairhurst’s artwork \textit{Gallery Connections}, noting that its high value and obsolescence require preserving the original artifact while replicating its function through other means \cite{laurenson_management_2005}. This pragmatic stance, while realistic, demands further reflection when applied to historical musical instruments already past the point of indefinite preservation.

Davies’ treatment of authenticity and substitution in musical performance \cite{davies_authenticity_2001} is influential but problematic. He draws a distinction between changing instrumentation in a way that constitutes a new transcription versus substitutions that do not, exemplified by the differences between harpsichord and pianoforte versus baroque and modern cello. Davies claims:

\begin{quotation}
    ``Where the change in instrumentation, and adaptations made to the music in the light of this, are significant, the result is a work transcription that is distinct from its model. One does not transcribe a work merely by crossing out the word ‘harpsichord’ on the score and replacing it with ‘piano’ (Davies 1988a), however. Neither does the substitution of the modern cello for its Baroque cousin result in a work transcription.''
\end{quotation}

This comparison conflates differences of kind and degree. The harpsichord and pianoforte share a keyboard but employ fundamentally different mechanisms for string excitation (plucking vs. hammering), representing a qualitative difference in instrument design and sound production. Conversely, the baroque and modern cello differ more in degree—both are bowed string instruments with broadly similar excitation physics, despite evolving dimensions and construction details.

This imprecise analogy suggests a partial misunderstanding of instrument classification and the embodied experience they generate. For example, within the keyboard family, harpsichords and fortepianos (early pianofortes) represent a gradual evolution with incremental changes in mechanism and expressive capacity, rather than a sharp dichotomy. Thus, any substitution or restoration must carefully consider how such changes affect musical and instructional values, rather than rely solely on categorical distinctions.

Instrument classification systems—such as the Hornbostel–Sachs taxonomy \cite{hornbostel_classification_1914}—group instruments by sound production methods, but this classification alone does not capture the nuances important for conservation decisions. Moreover, classification must be augmented with contextual knowledge about historical playing techniques, repertoire, and cultural practices to fully assess the impact of alterations.

This brings us back to the instructional value of instruments: their capacity to teach and communicate performance practice is deeply embedded in their material and functional characteristics. If substitutions or restorations alter these features, they risk severing vital links to past knowledge and experience.

\section{Case Study: Museo San Colombano Project}

Building upon these theoretical foundations, the Museo San Colombano in Bologna has embarked on an innovative project to enhance visitor engagement with historical keyboard instruments. The project includes designing a new exhibition space and creating an augmented replica of a historical harpsichord keyboard, enabling visitors to interact with and experience the instrument’s unique tactile and auditory qualities without risking damage to the original artifact.

This approach aligns with the emerging philosophy of living heritage and instructional value, recognizing that direct experience—whether through original instruments or faithful replicas—provides a crucial bridge between past and present. The augmented keyboard employs modern sensors and digital sound synthesis to replicate the physical feedback and acoustic response of the original, offering an immersive learning environment.

Initial visitor feedback and scholarly evaluation suggest that such technological mediation can effectively complement traditional conservation by enabling broader access to performative knowledge while safeguarding fragile originals. This project exemplifies a practical reconciliation between conservation imperatives and living heritage values.

\section{Conclusion}

The conservation of musical instruments requires a paradigm that embraces their dual identity as artifacts and living tools. By expanding conservation value frameworks to include instructional and embodied knowledge, we can better navigate the tensions between preserving material authenticity and enabling functional interaction. The Museo San Colombano project demonstrates that technological innovation can play a pivotal role in this reconciliation, fostering deeper, more meaningful engagement with musical heritage.

Future research should explore similar integrative approaches across other categories of heritage objects embodying craft and practice, further developing methodologies that honor both their historical materiality and living significance.

